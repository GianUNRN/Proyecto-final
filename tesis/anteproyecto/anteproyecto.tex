\documentclass[12pt,a4paper]{article}
\usepackage{fancyhdr}
\usepackage[utf8]{inputenc}
\usepackage[backend=biber, style=ieee]{biblatex}
\addbibresource{Bibliografia.bib} % Your bibliography file

\usepackage[spanish]{babel}
\usepackage{csquotes} % Recomendado por biblatex
\usepackage{hyperref} % Para URLs
\usepackage{url}
\usepackage{setspace}
\usepackage[margin=2.5cm,headheight=26pt,includeheadfoot]{geometry}

\usepackage{setspace}
\usepackage{graphicx}
\geometry{top=2.5cm, bottom=2.5cm, left=3cm, right=3cm}
\setstretch{1.3}

\begin{document}
	
		
		
	\thispagestyle{fancy}

	\lhead{\includegraphics[scale=1]{encabezado.png}}
	\rhead{Plan de Proyecto Final Integrador\\Ingenieria en Telecomunicaciones\\ Mangieri Gianfranco}
	\begin{flushright}
		San Carlos de Bariloche, \today
	\end{flushright}
	
	\vspace{1cm}
	
	\noindent
	A \\
	Jorge Cogo, Director de la Carrera de Ingeniería en Telecomunicaciones\\
	Universidad Nacional de Rio Negro.
%	Departamento de Estudiantes \\
%	Universidad Nacional de Río Negro \\
%	Sede Andina \\
	
%	\vspace{0.8cm}
%	
%	De mi mayor consideración:
	
	\vspace{0.5cm}
	
	
	
	
	Por la presente, me dirijo a usted con el fin de elevar el anteproyecto correspondiente al Trabajo Final de la carrera de Ingeniería en Telecomunicaciones para su revisión por parte del Consejo Asesor.
	
	Sin otro particular, quedo a disposición por cualquier observación o corrección que se considere pertinente.
	
	\vspace{2.5cm}
	
	Atentamente,
	
	
	Mangieri Gianfranco.
	
	
	

%	
%	Tengo el agrado de dirigirme a usted, en el marco establecido por el ``Protocolo General para la Elaboración de Trabajos Finales de carreras de la Sede Andina'', a fin de presentar el plan de trabajo propuesto para dar comienzo al desarrollo de mi \textbf{Proyecto Final Integrador} de la carrera de Ingeniería en Telecomunicaciones.  
%	
%	El tema a desarrollar es: \textbf{“Mejora de la detección en radares biestáticos pasivos basados en la señal de televisión digital ISDB-T a partir de la reconstrucción de la misma”}, el cual integra dos áreas centrales de la carrera: \textbf{Comunicaciones Inalámbricas} y \textbf{Procesamiento Digital de Señales}.  
%	
%	Se propone como director al \textbf{Dr. Javier Areta} (Profesor de la Carrera Ingeniería Electrónica, Sede Andina, UNRN – Docente/Investigador UNRN), quien avala esta presentación como firmante de la nota.  
%	
%	Sin más, lo saludo cordialmente,
%	
%	\vspace{2cm}
%	
%	\noindent
%	\textbf{Gianfranco Mangieri} \\
%	DNI: [completar] \\[0.5cm]
%	
%	\noindent
%	\textbf{Dr. Javier Areta} \\
%	Profesor, UNRN Sede Andina \\
%	DNI: [completar]
%	
	\newpage
%	
%	%=====================
	% CUERPO DEL ANTEPROYECTO
	%=====================
	
	\begin{center}
		\vspace*{1cm}
		
		
		
		
		\textbf{\MakeUppercase{Mejora de la detección  en radares biestaticos pasivos basados en la señal de televisión digital ISDB-T a partir de la reconstrucción de la misma }}
		
		
		
		
		
		\vspace{3.5cm}
		
		\textbf{Mangieri Gianfranco}
		
		
		
		\vspace{1.5cm}
		
		\textbf{Dr. Javier Areta}\\Director
		
		
		\vspace{2.5cm}
		
		
		\today
		
		
		\vspace{1.5cm}
		\Large
		Universidad Nacional De Rio Negro\\
		Argentina
		
	\end{center}
	\vspace{1cm}
	\newpage
	\section*{1 - Objetivos}
	
	\textbf{Objetivo general:}  
	Mejorar la capacidad de detección en radares biestáticos pasivos mediante el uso de la señal de Televisión Digital Terrestre (ISDB-T) como iluminador de oportunidad, a partir de la reconstrucción y procesamiento de dicha señal.
	
	\textbf{Objetivos específicos:}
	\begin{itemize}
		\item Analizar la geometría y fundamentos teóricos de los radares pasivos biestáticos.
		\item Estudiar la estructura de la señal ISDB-T y su potencial como iluminador de oportunidad.
		\item Implementar un modulador/demodulador ISDB-T en MATLAB para la reconstrucción de la señal.
		\item Desarrollar en Python la cadena de procesamiento radar: filtrado, procesamiento rango-doppler y detección.
		\item Incorporar la etapa de reconstrucción y remodulación de la señal ISDB-T (1-seg) para aprovechar la corrección de errores.
		\item Evaluar comparativamente el desempeño del sistema con y sin reconstrucción de la señal, tanto en simulaciones como en adquisiciones reales.
	\end{itemize}
	
	\section*{2 - Marco Teórico}
	
	Un sistema de radar biestático pasivo utiliza una señal externa, llamada \textbf{iluminador de oportunidad}, para realizar la detección de objetivos. El sistema consta de dos antenas:  
	\begin{itemize}
		\item La antena del \textbf{canal de referencia}, que capta la señal directa desde el transmisor ($x_r$).  
		\item La antena del \textbf{canal de vigilancia}, que capta la señal reflejada en un posible objetivo ($x_e$).  
	\end{itemize}
	
	A partir de estas dos señales, se aplican correlaciones cruzadas que permiten estimar:  
	\begin{itemize}
		\item El \textbf{retardo} entre $x_e$ y $x_r$, asociado a la distancia del blanco.  
		\item El \textbf{desplazamiento Doppler}, asociado a su velocidad radial.  
	\end{itemize}
	
	Un problema de este enfoque es que la señal de referencia real está contaminada por ruido y multitrayectoria, lo que degrada la SNR y reduce la probabilidad de detección.  
	
	En este proyecto, el iluminador será la \textbf{Televisión Digital Terrestre ISDB-Tb}, adoptada en Argentina a través de la resolución 7/2013 -- Anexo 1. Aunque ocupa un ancho de banda de 6–8 MHz, se trabajará únicamente con el segmento central \textbf{1-seg}, diseñado para receptores móviles. Esta decisión facilita la adquisición y reduce la complejidad de procesamiento.  
	
	El planteo del trabajo consiste en \textbf{reconstruir y remodular la señal de referencia}, aplicando la corrección de errores definida en la norma, con el fin de obtener una señal de mayor calidad para la correlación radar.
	
	\section*{3 - Modelo Base del Sistema}
	En la Figura \ref{model} se puede ver un diagrama ilustrativo de la cadena de procesamiento planteada en el proyecto.
		\begin{figure}[ht]
			\centering
			
			\includegraphics[width= .7\textwidth]{Procesamiento_pr.png}
			\caption{Modelo en bloques de prototipo del proyecto}
			\label{model}
		\end{figure}
		\newpage
		En donde cada bloque cumple las siguientes funcionalidades:
		\begin{itemize}
			\item Reconstrucción: demodulación y remodulación de la señal de referencia
			\item Filtro Clutter: filtro encargado de eliminar los ecos sin desplazamiento doppler, clutter, que enmascaran objetivos en movimiento.
			\item Cálculo Rango-Doppler: Procesamiento de correlación para estimar el rango y desplazamiento doppler de un posible objetivo
			\item Detección: Test de hipótesis para detección de objetivos  
		\end{itemize}
	\section*{4 - Factibilidad}
	
	El proyecto es factible de realizarse, ya que requiere únicamente de:
	\begin{itemize}
		\item Una PC con recursos de cómputo moderados.
		\item Herramientas de simulación y procesamiento de señales como MATLAB/Python.
		\item Medios de adquisición de datos, tales como una radio definida por software (SDR).
	\end{itemize}
	
	\section*{5 - Tareas y Cronograma}
	
	Para alcanzar los objetivos se propone el siguiente plan de trabajo:
	
	\begin{enumerate}
		\item Revisión bibliográfica sobre radares pasivos y la resolución 7/2013 -- Anexo 1 de la Enacom sobre la transmisión de TDT (Tv Digital Terrestre).
		
		\item Simulación de transmisor/receptor ISDB-T.
		\item Desarrollo de los bloques y algoritmos de procesamiento radar.

		\item Adquisición y procesamiento de datos reales. 
		\item Redacción del informe final y conclusiones.
	\end{enumerate}
	
	\textbf{Cronograma tentativo (en meses):}
	
	\begin{tabular}{|c|c|c|c|c|c|c|}
		\hline
		Tarea & 1 & 2 & 3 & 4 & 5 & 6 \\
		\hline
		Revisión bibliográfica & X & X & X  &   &   &      \\
		
		Simulación ISDB-T &   &  X & X & X & X  &     \\
		Desarrollo de procesamiento radar. & & & X & X & X & X \\ 
		Adquisición de datos &   &   &   &   &  X & X    \\
		Redacción informe final &   &   &   &   &   &    X \\
		\hline
	\end{tabular}
	
	
	\cleardoublepage
	\addcontentsline{toc}{chapter}{Bibliografía}
	\nocite{isdbt}
	\nocite{malanowski}
	\nocite{richards2014radar}
	\printbibliography
	
	
\end{document}
