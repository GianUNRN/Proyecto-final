\documentclass[12pt, letterpaper, oneside]{report}
\usepackage[utf8]{inputenc}
\usepackage[T1]{fontenc}
\usepackage{microtype}
\usepackage{mathptmx} % Times New Roman-like font
\usepackage{setspace}
\usepackage[top=1in, bottom=1in, left=1.5in, right=1in]{geometry}
\usepackage[spanish]{babel}
\usepackage{csquotes}
\usepackage{makeidx}

\usepackage{tabularx}
\newcolumntype{b}{>{\hsize=1.5\hsize}X}
\newcolumntype{s}{>{\hsize=.5\hsize}X}

\usepackage[backend=biber, style=ieee]{biblatex}
\addbibresource{Bibliografia.bib} % Your bibliography file

% Chapter formatting
\usepackage{titlesec}
\titleformat{\chapter}[display]
{\normalfont\Large\bfseries}{\chaptertitlename\ \thechapter}{20pt}{\LARGE}
\titlespacing*{\chapter}{0pt}{-30pt}{40pt}

% Table of contents formatting
\usepackage[titles]{tocloft}
\renewcommand{\cftchapfont}{\normalfont\bfseries}
\renewcommand{\cftchapleader}{\cftdotfill{\cftdotsep}}
\addto\captionsspanish{\renewcommand{\tablename}{Tabla}}

% Figure and table captions
\usepackage{float}
\usepackage[font=small, labelfont=bf]{caption}
\usepackage{graphicx} % Include images
\graphicspath{ {./figuras/} }

\usepackage{booktabs}

% Mathematics packages
\usepackage{amsmath}
\usepackage{amssymb}
\usepackage{amsthm}

% Theorems and definitions
\newtheorem{theorem}{Theorem}[chapter]
\newtheorem{lemma}[theorem]{Lemma}
\newtheorem{proposition}[theorem]{Proposition}
\newtheorem{corollary}[theorem]{Corollary}
\theoremstyle{definition}
\newtheorem{definition}[theorem]{Definition}
\newtheorem{example}[theorem]{Example}

% Header and footer
\usepackage{fancyhdr}
\pagestyle{fancy}
\fancyhf{}
\fancyhead[L]{\leftmark}
\fancyfoot[C]{\thepage}
\renewcommand{\headrulewidth}{0.4pt}
\renewcommand{\footrulewidth}{0pt}


% Line spacing
\onehalfspacing



\newcommand \imagen[2]{\centerline{\includegraphics[scale=#1]{#2}}}
\newcommand \p[1] {\hspace{1pt}\par\vspace{#1 pt}}
\newcommand \figura[5]{
	\begin{figure}[H]
		\imagen{#1}{#2}\p{5}
		\caption[#3]{#3 #4}
		#5
	\end{figure}
}


\begin{document}
	
	% Title page
	\begin{titlepage}
		\begin{center}
			\vspace*{1cm}
			\LARGE
			PROYECTO INTEGRADOR DE LA CARRERA DE INGENIERÍA EN TELECOMUNICACIONES
			
			
			\vspace{1.5cm}
			
		
			
			
			
			\textbf{\MakeUppercase{Mejora de la detección  en radares biestaticos pasivos basados en la señal de televisión digital ISDB-T a partir de la reconstrucción de la misma }}
			
		
			
			
			
			\vspace{1.5cm}
			
			\textbf{Mangieri Gianfranco}
			
			
			
			\vspace{1.5cm}
			
			\textbf{Dr. Javier Areta}\\Director
			
			
			\vspace{1.5cm}
			
			
			\today
			
			
			\vspace{1.5cm}
			\Large
			Universidad Nacional De Rio Negro\\
			Argentina
			
		\end{center}
	\end{titlepage}
	

	
	% Dedication (optional)
	\cleardoublepage
	\thispagestyle{empty}
	\vspace*{\fill}
	\begin{center}
		\textit{To my family and mentors...}
	\end{center}
	\vspace*{\fill}
	\cleardoublepage
	

	\cleardoublepage
	\chapter*{Acronimos}
	\addcontentsline{toc}{chapter}{Acronimos}
		\begin{tabularx}{\textwidth}{sb}
			\textbf{16QAM} & Modulación de amplitud en cuadratura de 16 estados, por sus siglas en inglés.\\
			\textbf{64QAM} & Modulación de amplitud en cuadratura de 64 estados, por sus siglas en inglés.\\
			\textbf{BW} & Ancho de banda, por sus siglas en inglés.\\
			\textbf{CAF}& Función de ambigüedad cruzada, por sus siglas en inglés.\\
			\textbf{COFDM}& Multiplexación por división de frecuencias ortogonales codificada, por sus siglas en inglés.\\
			\textbf{DBPSK}& Modulación por desplazamiento de fase binaria diferencial, por sus siglas en inglés.\\
			\textbf{FEC} & Corrección de errores hacia adelante, por sus siglas en inglés.\\
			\textbf{FFT} & Transformada rápida de Fourier, por sus siglas en inglés.\\
			\textbf{IFFT} & Transformada rápida de Fourier inversa, por sus siglas en inglés.\\
			\textbf{IQ} & En fase y cuadratura, por sus siglas en inglés.\\
			\textbf{ISDB-T} & Radiodifusión Digital de Servicios Integrados - Terrestre, por sus siglas en inglés.\\
			\textbf{OFDM} & Multiplexación por división de frecuencias ortogonales, por sus siglas en inglés.\\
			\textbf{PRBS} & Secuencia binaria psudo-random, por sus siglas en inglés.\\
			\textbf{QAM} & Modulación de amplitud en cuadratura, por sus siglas en inglés.\\
			\textbf{QPSK} & Modulación por desplazamiento de fase cuaternaria, por sus siglas en inglés.\\
			\textbf{RCS} & Sección equivalente de radar, por sus siglas en inglés.\\
			\textbf{RS} & Reed-Solomon.\\
			\textbf{SNR} & Relación señal a ruido, por sus siglas en inglés.\\
			\textbf{TDT} & Televisión digital terrestre.\\
			\textbf{TMCC} & [completar].\\
			\textbf{CFAR} & [completar]\\
			\textbf{DSI} &  [completar]\\			
		\end{tabularx}
		
	
	\cleardoublepage
	
	
	% Table of Contents
	\tableofcontents
	\cleardoublepage
	
	% List of Figures
	\listoffigures
	\addcontentsline{toc}{chapter}{Índice de figuras}
	\cleardoublepage
	
	% List of Tables
	
	\renewcommand*\listtablename{Índice de tablas}
	\listoftables
	\addcontentsline{toc}{chapter}{Índice de tablas}
	\cleardoublepage
	
		% Abstract
	\cleardoublepage
	\addcontentsline{toc}{chapter}{Resumen}
	\begin{abstract}
		
		
		\vspace{0.5cm}
		
		\doublespacing
		This is where you write your abstract. It should provide a concise summary of your research, including the problem statement, methodology, key findings, and conclusions. Typically, an abstract is between 150-350 words.
		
		\textbf{Keywords:} keyword1, keyword2, keyword3, keyword4
	\end{abstract}
	\cleardoublepage

%	\include{capitulos/introduccion}
%	
\chapter{Radar Pasivo}
	El radar pasivo se diferencia del radar activo en que no cuenta con un transmisor propio, sino que utiliza fuentes externas de iluminación, como emisores de telecomunicaciones. La detección de objetivos se realiza comparando la señal directa del transmisor con la señal reflejada por el blanco.
	
	Para este fin, el sistema dispone al menos de dos canales de recepción: referencia y eco. El canal de referencia obtiene la señal original transmitida orientando una antena direccional hacia la fuente emisora o generando digitalmente un haz en esa dirección. El canal de eco, en cambio, capta la señal reflejada desde la zona bajo vigilancia mediante otra antena o un haz formado digitalmente.
	
	El procesamiento conjunto de ambas señales permite detectar los objetivos. La medición principal en radar pasivo es el alcance biestático, definido como la diferencia entre la distancia transmisor–objetivo–receptor y la distancia transmisor–receptor. Los puntos con igual alcance biestático se ubican sobre un elipsoide cuyos focos son el transmisor y el receptor.
	
	Además, el radar pasivo mide el desplazamiento Doppler, proporcional a la velocidad biestática, es decir, la variación temporal del alcance biestático. La localización de objetivos en coordenadas cartesianas puede obtenerse calculando el punto de intersección de varios elipsoides biestáticos generados por diferentes pares transmisor–receptor. Alternativamente, la posición del blanco puede estimarse combinando la información de alcance biestático con la dirección de llegada de la señal, un método similar al que emplea el radar monostático clásico
	
	
	\section{Geometría y coordenadas biestáticas}
	
	Consideremos la geometría de un radar pasivo en coordenadas cartesianas. Un blanco se encuentra en la posición \((x(t), y(t), z(t))\); el transmisor está ubicado en \((x_{t}, y_{t}, z_{t})\) y el receptor en \((x_{r}, y_{r}, z_{r})\). Esto no permite definir los siguientes rangos: el rango transmisor-objetivo ($R_1$), el receptor-objetivo ($R_2$) y el rango transmisor-receptor ($R_b$).
	\begin{equation}
		R_1(t) = \sqrt{(x(t) - x_t)^2 + (y(t) - y_t)^2  +(z(t) - z_t)^2}
	\end{equation}
	
	\begin{equation}
	R_2(t) = \sqrt{(x(t) - x_r)^2 + (y(t) - y_r)^2  +(z(t) - z_r)^2}
	\end{equation}
	
	\begin{equation}
	R_b = \sqrt{(x_t - x_b)^2 + (y_t - y_b)^2  +(z_t - z_b)^2}
	\end{equation}
	
	En donde el ángulo entre los segmentos transmisor–objetivo y receptor–objetivo, denotado por \(\beta\), se conoce como ángulo biestático.
	\figura{.4}{./pr/geometria.png}{Geometría en un esquema biestático.}{ Fuente: \cite{malanowski}.}{\label{fig:geo}}
	
	
	
	El rango biestático se calcula a partir de la diferencia entre el camino directo, $R_b$, y el indirecto, $R_1(t) + R_2(t)$. Esto puede hacerse a partir del retardo \(\tau\) medido entre las señales de eco y de referencia, y la velocidad de la luz \(c\).
	
	\begin{equation}
		R(t) = R_1(t) - R_2(t) - R_b = c \cdot \tau
		\label{eq:rango_biestatico}
	\end{equation}
	
	El lugar geométrico de los puntos con igual rango biestático forma un elipsoide biestático en el espacio tridimensional.  
	Los focos de este elipsoide se encuentran en las posiciones del transmisor y del receptor.  
	La intersección bidimensional de este elipsoide con un plano que contiene al transmisor y al receptor forma una elipse biestática.
	\figura{.6}{./pr/rango_bi_3d.png}{Elipsoide biestático de iso-rango.}{ Fuente: \cite{malanowski}.}{\label{fig:rango_3d}}
	\figura{.6}{./pr/rango_bi.png}{Elipses biestáticas.}{ Fuente: \cite{malanowski}.}{\label{fig:rango}}
	
	
	
	En la Figura~\ref{fig:rango} se presenta un ejemplo de una familia de elipses biestáticas, en donde  rango biestático creciente corresponde a elipses biestáticas de mayor tamaño.
	
	
	Además del rango biestático, otro parámetro que se mide rutinariamente con el radar pasivo es la velocidad biestática.  
	Esta se define como la derivada temporal del rango biestático.  
	
	La velocidad biestática se calcula a partir del desplazamiento Doppler \(f_{d}\) medido entre las señales de referencia y de eco, y de la longitud de onda \(\lambda = \frac{c}{f_{c}}\), donde \(f_{c}\) es la frecuencia portadora:
	
	\begin{equation}
		V = \dfrac{\partial R}{\partial t} = -\lambda f_d
		\label{eq:vel_biestatica}
	\end{equation}
	
	
		
		
	\section{Ecuación Radar}
	
	 Se ilustra el esquema básico. La distancia entre el transmisor y el objetivo es $R_{1}$, mientras que la distancia entre el objetivo y el receptor es $R_{2}$. El transmisor emite una potencia $P_{t}$.
	
	Si la radiación fuera isotrópica, es decir, con igual intensidad en todas las direcciones, la densidad de potencia a una distancia $R_{1}$ sería igual a la potencia total emitida dividida por el área de la esfera de radio $R_{1}$, es decir $4\pi R_{1}^{2}$. Sin embargo, la directividad de la antena transmisora $G_{t}$ multiplica la densidad de potencia en la dirección de máxima radiación. Suponiendo que el objetivo se encuentra a la distancia $R_{1}$ en esa dirección, la densidad de potencia incidente sobre el blanco puede expresarse como:
	
	\begin{equation}
		S_{1}=\frac{P_{t} G_{t}}{4\pi R_{1}^{2}}
		\label{eq:S1}
	\end{equation}
	
	La cantidad de potencia reflejada por el objetivo está determinada por su sección eficaz radar (RCS, por sus siglas en inglés) $\sigma$. Esta magnitud, expresada en unidades de área (por ejemplo, m$^{2}$), se define como:
	
	\begin{equation}
		\sigma = \lim_{R\to\infty}\frac{4\pi R^{2} S_{s}}{S_{i}}
		\label{eq:sigma}
	\end{equation}
	
	donde $S_{s}$ es la densidad de potencia dispersada y $S_{i}$ es la densidad de potencia incidente.
	En este modelo teórico se asume que la potencia dispersada se irradia isotrópicamente. En la práctica esta condición no siempre se cumple, pero resulta útil para definir $\sigma$. En la ecuación anterior, $4\pi R^{2}$ es el área de una esfera de radio $R$; al multiplicarla por la densidad dispersada $S_{s}$ se obtiene la potencia total reflejada por el blanco. Así, la RCS representa la relación entre la potencia dispersada y la densidad de potencia incidente.
	
	En un radar pasivo, la RCS relevante es la biestática, pues la onda incidente y la onda dispersada viajan en direcciones distintas, correspondientes al transmisor y al receptor respectivamente.
	
	Asumiendo que la densidad de potencia incidente es $S_{1}$, la densidad de potencia reflejada a la distancia $R_{2}$ desde el blanco, con una RCS biestática $\sigma$, es:
	
	\begin{equation}
		S_{2}=\frac{P_{t} G_{t} \sigma}{(4\pi)^{2} R_{1}^{2} R_{2}^{2}}
		\label{eq:S2}
	\end{equation}
	
	La potencia recibida por el radar se determina a partir de la densidad de potencia $S_{2}$ y del área efectiva de la antena receptora $A_{\text{ef}}$:
	
	\begin{equation}
		P_{r}=\frac{P_{t} G_{t} \sigma A_{\text{ef}}}{(4\pi)^{2} R_{1}^{2} R_{2}^{2}}
		\label{eq:PrAef}
	\end{equation}
	
	Habitualmente, la antena receptora se caracteriza mediante su ganancia $G_{r}$, la cual se relaciona con el área efectiva de acuerdo con:
	
	\begin{equation}
		A_{\text{ef}}=\frac{G_{r}\lambda^{2}}{4\pi}
		\label{eq:Aef}
	\end{equation}
	
	Sustituyendo esta expresión en la ecuación~\eqref{eq:PrAef}, la potencia de eco recibida por el radar resulta:
	
	\begin{equation}
		P_{r}=\frac{P_{t} G_{t} \sigma G_{r} \lambda^{2}}{(4\pi)^{3} R_{1}^{2} R_{2}^{2}}
		\label{eq:PrFinal}
	\end{equation}
	

	\section{Iluminadores de Oportunidad}
	
	A diferencia de un sistema de radar convencional, en un radar biestático pasivo no generamos la señal de referencia, pero podemos elegir cuál utilizar. Dependiendo de la aplicación, cada señal tendrá sus ventajas y desventajas. Los dos principales parámetros para seleccionar el iluminador son la frecuencia y el ancho de banda.
	
	Al observar la ecuación de resolución en rango, podemos ver que aumentar el ancho de banda reduce el tamaño de la celda de resolución en rango; además, al incrementar la frecuencia también aumentan las pérdidas por propagación en el espacio libre, dadas por
	
	\begin{equation}
		\label{eq:pel}
		\alpha_{PEL} = \left(\frac{4 \pi d f}{c}\right)^2
	\end{equation}
	
	donde \(f\), \(d\) y \(c\) representan la frecuencia, la distancia recorrida por la señal y la velocidad de la luz, respectivamente. Esto determina el rango máximo de detección antes de que la señal se atenúe por completo.
	
	Algunas de las señales más comunes para este uso son:
	
	\begin{table}[H]
		
		\begin{tabular}{l l l l}
			\hline
			Iluminador & Banda de frecuencia & BW & P$_{t}$ \\
			\hline
			DVB-T & 470--860 MHz & 7.6 MHz & Máx. 100 kW \\
			GSM & 935--960 MHz & 200 kHz & Máx. 320 W \\
			LTE & 700 MHz--2.6 GHz & 1.4--20 MHz & Máx. 200 mW \\
			WiFi & 2.4 GHz--5 GHz--6 GHz & 16 MHz & Máx. 200 mW \\
			\hline
		\end{tabular}
		\centering
		\caption[Ejemplos de iluminadores para radar pasivo.]{Ejemplos de iluminadores para radar pasivo. Adaptado de \cite{malanowski}.}
		\label{iluminadores}
	\end{table}
	
	En este trabajo se eligió la señal de TDT. Uno de los motivos es que su estructura y procesamiento están estrictamente definidos, lo que permite conocer exactamente cómo la estación transmisora emitió la señal y qué pasos seguir para demodularla y volverla a modular. Sin embargo, un problema intrínseco de esta señal es la aparición de lóbulos laterales determinísticos debido a sus prefijos cíclicos y herramientas de sincronización (véase Capítulo~4).
	
	Si bien originalmente la señal de TDT cuenta con un ancho de banda de aproximadamente 8 GHz, utilizar el espectro completo complejiza notablemente el modulador y el demodulador, por lo que se optó por emplear el segmento central de transmisión parcial. Este cuenta con un ancho de banda de [completar], el cual es menor pero suficiente para aplicarlo como base de trabajo.
	

	\section{Canal de referencia y canal de vigilancia}
	
	
	Tomando a la señal de referencia como
	
	\begin{equation}
		\label{eq:env_compeja_xr}
		x_r^{RF}(t) = \mathfrak{R}\left\lbrace x_r(t) \cdot exp(j2\pi f_ct)\right\rbrace
	\end{equation}
	en donde $x_r^{RF}(t)$ es la señal pasabanda real, $f_c$ es la frecuencia central y $x_r(t)$ es su envolvente compleja. Podemos definir a nuestra señal eco como
	
	
	\begin{equation}
		\label{eq:env_compeja_xe}
		x_e^{RF}(t) = \mathfrak{R}\left\lbrace C \cdot x_r(t-\frac{r(t)}{c}) \cdot exp(j2\pi f_c(t - \frac{r(t)}{c}))\right\rbrace
	\end{equation}
	
	Teniendo en cuenta que podemos expandir a r(t) usando la serie de taylor
	\begin{equation}
		r(t) = \sum_{k=0}^\infty \dfrac{r^{(k)}(t)}{k!}t^k = R + Vt + A\frac{t^2}{2}+ \cdots 
	\end{equation}
	podemos entonces aproximarlo como
	\begin{equation}
		\label{eq:aprox_rango}
		r(t) \approx R + Vt
	\end{equation}
	
	Usando la ecuación \ref{eq:aprox_rango}, la envolvente compleja de la señal eco queda
	
	
	\begin{equation}
		\label{eq:aprox_env_xe}
		x_e(t) \approx  C'exp(j\frac{2\pi}{\lambda}R) \cdot x_r(t - \frac{R}{c})exp(j\frac{2\pi}{\lambda}Vt) )
	\end{equation}
	con C$'$ es la amplitud compleja del eco	
	
	Idealmente, cada canal debería estar aislado del otro y las señales no deberían contener componentes de los canales opuestos. Sin embargo, esto representa un problema ya que, por más directivas que sean las antenas, siempre se filtrará cierta cantidad de potencia. 
	\figura{1}{./pr/canales.png}{Correlación entre antenas de vigilancia y referencia.}{ Fuente: \cite{malanowski}.}{\label{fig:canales}}
	En la Figura~\ref{fig:canales} se observa cómo una componente de la señal eco, $A_{surv}(\phi_{tx})$, es captada por el diagrama de radiación de la antena de vigilancia (línea sólida) y, de forma inversa, la señal de referencia, $A_{ref}(\phi_{target})$, es recibida por la antena de vigilancia (línea punteada). 
	
	En el caso del canal de referencia, esta interferencia no implica una gran diferencia, dado que la potencia del eco es despreciable. Sin embargo, lo mismo no ocurre en sentido inverso, ya que sin un tratamiento previo la señal directa enmascara completamente al objetivo. Este fenómeno se conoce como \textit{interferencia de señal directa} (DSI, por sus siglas en inglés) y debe tratarse adecuadamente junto con los rebotes sin desplazamiento Doppler, es decir, el \textit{clutter}.  
	
	De forma análoga a un radar convencional, se aplica un filtro anti-\textit{clutter} encargado de eliminar de la señal eco todos los componentes con $f_{\text{doppler}}$ nula. La selectividad con la que se suprimen estos componentes depende del tipo de filtro y de la cantidad de coeficientes empleados.
      
	
	\figura{.6}{./pr/filtro.png}{Diagrama de un filtro anti clutter.}{}{\label{fig:filtro_clutter}}
	
	\chapter{Transmisión de Señal de TV Digital Terrestre ISDB-T}


En Argentina, la transmisión de TDT está regulada por el ENACOM a través de la resolucion 7/13. Esta establece el esquema ISDB-Tb definido por la norma ABNT NBR 15601. Además, en el Anexo 1 se establece las especificaciones técnicas, tales como el tratamiento de datos, la modulación y la transmisión. En este documento también se definen los modos de recepción, especificándose la recepción total o parcial, denominada 1-seg. Asimismo, la normativa describe cómo se divide el espectro de un canal de 8 MHz en 13 segmentos, siendo el segmento central el destinado a la recepción parcial. Además, en la Figura X se presenta la cadena completa de procesamiento.


	En la Figura \ref{fig:bloques_isdb} puede observarse todos los bloques de procesamiento aplicado en un transmisor de TDT. Este capitulo se encarga de explicar cada uno de ellos en el contexto de transmisión y recepción parcial 1-seg. 
	
	\figura{.5}{./isdb/orden_canales.png}{Orden de los segmentos en el canal de 8 MHz.}{ Fuente: \cite{isdbt}.}{\label{fig:canales_isdb}}
	
	\figura{.5}{./isdb/1seg.png}{Recepción total y parcial.}{ Fuente: \cite{isdbt}.}{\label{fig:1seg}}
	
	
	\figura{.7}{./isdb/cod_bloques.png}{Codificación de canal.}{ Fuente: \cite{isdbt}}{\label{fig:bloques_isdb}}
	
	
	
	\section{Dispersión de energía}
	Luego del tratamiento de los *transport streams*, el primer bloque es el dispersor de energía. Para ello se utiliza una secuencia pseudoaleatoria que modula los bits. Esto se hace a partir de un shift register retroalimentado, el cual genera la secuencia, y una compuerta XOR que invierta los datos en función de la secuencia. La configuración inicial del generador esta dada por la semilla [] y el polinomio [], como se muestra en la Figura X. 
	
	\section{Código RS}
	Una vez dispersada la energía, se aplica el primer nivel de codificación. Se utiliza un código RS ( ) con capacidad para detectar y corregir hasta **x** errores. Para esto se utiliza el polinomio generador []. 
	\section{Entrelazador de byte}
	
	Posteriormente, los datos se entrelazan a nivel de byte mediante un interleaver convolucional, consistente en una cascada de shift registers con distintos niveles de profundidad, tal como se muestra en la Figura X, de manera que las distintas ramas sufran distintos retardos y así generar el entrelazaminento.
	
	
	\figura{.7}{./isdb/intrlv_conv.png}{Entrelazador de bytes.}{ Fuente: \cite{isdbt}.}{\label{fig:intrlv_conv}}
	
	
	\section{Código Convolucional}
	
	A continuación, comienza el segundo nivel de codificación. En esta etapa se emplea un código convolucional puntuado, cuyos polinomios característicos son [ ] y [ ] para las señales X e Y. Dependiendo del patrón de puntuado, la combinación correspondiente de estas señales se utilizará como salida. En la Tabla X se detallan las combinaciones asociadas a cada patrón.
	
	
	\figura{.7}{./isdb/conv_code.png}{Diagrama en bloques del código convolucional.}{ Fuente: \cite{isdbt}.}{\label{fig:conv_code}}
	
	\section{Entrelazador de bit}
	Análogamente al entrelazador de bytes, se utiliza un interleaver convolucional en donde el tamaño del registro es un bit y la cantidad de ramas y retardos, dados por la profundidad de los registros, depende del modo y modulación de operación.
	
	
	\section{Mapeo de símbolos QAM}
	
	Una vez completado el segundo nivel de codificación, se aplica una segunda etapa de entrelazado, esta vez a nivel de bit. De manera análoga a la anterior, se utiliza un *interleaver* convolucional, aunque en este caso cada registro almacena únicamente un bit. Finalmente, los bits se mapean a símbolos de una de las tres modulaciones QAM disponibles: QPSK (grupos de 2 bits), 16-QAM (grupos de 4 bits) y 64-QAM (grupos de 6 bits). En todos los casos, los bits pares e impares representan la parte real e imaginaria, respectivamente, siguiendo el mapeo especificado.
	
	\subsubsection{QPSK}
	\figura{.7}{./isdb/qpsk.png}{Mapeo de bits a símbolos QPSK.}{ Fuente: \cite{isdbt}.}{\label{fig:qpsk}}
	\begin{equation}
		a_k = 1 - 2  b_0  +j - 2j  b_1
	\end{equation}
	
	
	\subsubsection{16 QAM}
	\figura{.7}{./isdb/16qam.png}{Mapeo de bits a símbolos 16 QAM.}{ Fuente: \cite{isdbt}.}{\label{fig:16qam}}
	
	\begin{align}
		\mathbb{\textbf{R}}\lbrace a_k \rbrace &= 3 - 6 b_0 - 2  b_2 + 4  b_0  b_2\\
		\mathbb{\textbf{I}}\lbrace a_k \rbrace &= 3 - 6 b_1 - 2  b_3 + 4  b_1  b_3 
	\end{align}
		
	
	
	\subsubsection{32 QAM}
	\figura{.9}{./isdb/32qam.png}{Mapeo de bits a símbolos 32 QAM.}{ Fuente: \cite{isdbt}.}{\label{fig:32qam}}
	\begin{align}
		\mathbb{\textbf{R}}\lbrace a_k \rbrace &= 7 -14  b_0 - 6  b_2 - 2  b_4 + 12 ¿ b_0  b_2 + 4  b_0  b_4 + 4  b_2  b_4 - 2  b_0  b_2  b_4\\
		\mathbb{\textbf{I}}\lbrace a_k \rbrace &= 7 -14  b_1 - 6  b_3 - 2  b_5 + 12  b_1  b_3 + 4  b_1  b_5 + 4  b_3  b_5 - 2  b_1  b_3  b_5
	\end{align}
	
	\section{Entrelazador de tiempo y frecuencia}
	\section{Estructura de frame OFDM}
	\figura{.7}{./isdb/frame.png}{Estructura de datos en el frame OFDM.}{ Fuente: \cite{isdbt}.}{\label{fig:frame}}
	\subsubsection{Pilotos y PRBS}
	\figura{.7}{./isdb/pilotos.png}{Distribucion de portadoras pilotos.}{ Fuente: \cite{isdbt}.}{\label{fig:pilotos}}
	
	\figura{.7}{./isdb/prbs.png}{Generador de secuencia binaria pseudo aleatoria .}{ Fuente: \cite{isdbt}.}{\label{fig:prbs}}
	\section{Frecuencia de muestreo y frecuencias centrales}
	
\chapter{Recepcion}
Para la recepcion de una señal ISDBT se implean los siguientes bloques de procesamiento

En primer lugar es necesario identificar los simbolos OFDM, esto se hace a traves del prefijo ciclico agergado en la etapa de tranmision. Este al ser una fraccion de la parte final del mensaje puede identificarse
usando una correlacion. Al tomar X cantidad de muestras podemos obtener el incio del simbolo como el pico maximo de la correlacion sumado a la duracion del CP. 

Una vez identificado el simbolo, el siguiente paso es corregir el error de fase presente en la señal. Esto puede hacerse tambien a partir del CP
viendo la diferencia de fase entre este (en el incio del mensaje) y su parte correspondiente en el payload (en el final del mensaje). 
Sin embargo, esta correccion no es suficiente por si misma. Para terminar de centrar el espectro y ubicar la posicion de las 
distintas portadoresa se utilizan las portadoras piloto. Al ser estas una secuencia conocida y en posiciones predeterminadas,
pueden ser usadas con una correlacion para encontrar la poscicion de las portadoras usando
[eq]
en donde el pico maximo indicara que la secuencia conocida se alineo con su contraparte dentro de los datos del simbolo OFDM. 
Ya con esto pueden extraerse los datos de control y de transmision y proceder con la demodulacion de manera acorde.
%	\chapter{Cadena de procesamiento radar}
	
	\section{Filtro de clutter y rayo directo}

	La detección de objetivos en un radar pasivo se ve limitada, en gran medida, por la presencia de la interferencia de trayectoria directa (ITD) y por las reflexiones de clutter. Ambos componentes suelen dominar la señal recibida y generan lóbulos secundarios en la función de ambigüedad cruzada que pueden ocultar ecos débiles, aun cuando éstos se encuentren por encima del nivel de ruido del receptor. Este problema es especialmente crítico si se considera que, para los valores típicos del producto *BT* utilizados en aplicaciones reales, las fluctuaciones residuales pueden situarse varios decibelios por encima de los ecos de interés.
	
	En este contexto, el filtrado adaptativo se convierte en una herramienta fundamental para mejorar el rango dinámico efectivo del sistema. La señal de vigilancia está compuesta por la contribución directa del transmisor, por múltiples reflexiones estacionarias de distinto retardo (clutter), por los ecos de los objetivos en movimiento y por ruido. La ITD suele ser el componente de mayor potencia, seguida por el clutter, cuyas copias retardadas de la señal de referencia pueden estar entre 20 y 40 dB por debajo de la trayectoria directa. Ambos, no obstante, se mantienen muy por encima del ruido térmico o del ruido antropogénico dominante en bandas UHF y VHF, generando un enmascaramiento que debe ser reducido mediante procesamiento digital.
	
	Si bien existen métodos físicos o analógicos para mitigar estas interferencias, como el blindaje de antenas o la cancelación en RF, su efectividad es limitada. Por este motivo, prácticamente todas las implementaciones modernas recurren a técnicas digitales de cancelación. Entre ellas se incluyen enfoques iterativos basados en filtros adaptativos —como LMS, NLMS, RLS o estructuras en celosía— y métodos de procesamiento por bloques. Cada familia de algoritmos presenta ventajas y desventajas en términos de velocidad de convergencia, estabilidad, complejidad computacional y capacidad para discriminar componentes estrechamente espaciados en frecuencia. También existen métodos afines al filtrado adaptativo tradicional, como el algoritmo CLEAN, que permiten eliminar componentes dominantes de forma secuencial.
	
	El propósito de este capítulo es presentar el problema de la cancelación de clutter e ITD desde un marco unificado, describir los principales algoritmos utilizados para tal fin y analizar su comportamiento en escenarios reales. Se introduce un modelo de señal adecuado para formular el problema del filtrado adaptativo y se discute cómo las características de los ecos —tanto estacionarios como móviles— afectan la capacidad de cancelación. Finalmente, se muestran resultados obtenidos con señales reales y se examinan aspectos prácticos como la convergencia y la selectividad espectral de los filtros.
	
		\subsection{Filtro de lattice en bloque}
		
		
		El filtro \textit{block lattice} se basa en una serie de proyecciones sucesivas entre señales, cuya dinámica queda determinada por los coeficientes de reflexión $\kappa_i$. Estos coeficientes pueden interpretarse como proyecciones normalizadas entre el error de predicción hacia adelante y el error de predicción hacia atrás en cada etapa del filtro. En términos geométricos, el cálculo de $\kappa_i$ equivale a medir cuánta energía de una señal puede explicarse mediante la otra, por lo que constituye una operación análoga al proceso de ortogonalización de Gram--Schmidt aplicado en un espacio vectorial de señales.
		
		Como resultado de estas proyecciones, los errores de predicción hacia atrás $b_i(n)$ se vuelven mutuamente ortogonales. Cada nuevo error contiene únicamente la parte de la señal que no pudo predecirse a partir de las etapas anteriores, garantizando así que la información aportada por cada $b_i(n)$ sea \emph{nueva} en el sentido estadístico. Esta propiedad de ortogonalidad es fundamental para la capacidad del filtro de separar componentes correlacionadas sin interferencia entre ellas.
		
		La ortogonalidad implica que cada $b_i(n)$ está asociado a un subespacio distinto y no superpuesto. Esto permite que las copias retardadas de la señal de referencia presentes en la señal de eco $x_e(n)$ puedan ser tratadas de manera independiente. Para cada etapa, se estima un coeficiente de correlación $h_i$ calculado como la proyección de $x_e(n)$ sobre el error de predicción hacia atrás. Este coeficiente determina cuánta contribución de $b_i(n)$ está presente en el eco.
		
		El proceso de cancelación del eco se realiza mediante la resta
		\[
		e_{i+1}(n) = e_i(n) - h_i\, b_i(n),
		\]
		que elimina únicamente la componente de $x_e(n)$ que está alineada con $b_i(n)$. Debido a la ortogonalidad entre los $b_i(n)$, esta operación no afecta a ninguna otra componente del eco. En consecuencia, la señal remanente conserva intacta toda porción que no esté correlacionada con la referencia, preservando así las reflexiones asociadas a objetivos reales.
		
		En términos geométricos, el filtro \textit{block lattice} realiza una secuencia de sustracciones ortogonales: en cada etapa, se extrae de $x_e(n)$ la proyección correspondiente a una copia retardada de la referencia. Esto permite remover el \textit{clutter} de forma progresiva y precisa, sin introducir distorsiones ni interferencias cruzadas entre etapas. Esta propiedad lo convierte en una herramienta especialmente adecuada para aplicaciones de radar pasivo, donde el eco contiene múltiples réplicas del pulso directo con distintos retardos y amplitudes.
		
		\figura{.5}{./pr/lattice.png}{Estructura de filtro en bloque de Lattice.}{ Fuente: \cite{malanowski}}{\label{fig:lattice}}
		
		En donde el algoritmo de procesamiento esta dados las ecuaciones de la Tabla \ref{table:lattice}.
		
		
		\begin{table}[H]
			
			\centering
			\begin{tabular}{|l|c|}
				\hline
				&\\
				&
				$k_{i+1} = 2 \cdot \dfrac{< b_i(n-1); f_i(n)>}{|| b_i(n-1)||^2 + || f_i(n)||^2} $\\
				Predicción&\\
				&$b_{i+1}(n) = b_i(n-1) - k_{i+1} \cdot f_i(n)$\\
				&\\
				&$f_{i+1}(n) = f_i(n) - k_{i+1}^\ast \cdot  b_i(n-1)$\\\hline
				 &\\
				
				Filtrado&$h_i = \dfrac{<e_i(n); b_i(n)> }{|| b_i(n)||^2} $\\
				&\\
				&$e_{i+1}(n) = e_i(n) - b_i(n) \cdot h_i$\\\hline
			\end{tabular}
			\centering
			\caption[Algoritmo de filtro lattice en bloque.]{Algoritmo de filtro lattice en bloque. Adaptado de \cite{malanowski}.}
			\label{table:lattice}
			
		\end{table}
		
		En donde $<\bullet \hspace{3pt}; \bullet>$ representa el producto interno característico de $\mathbb{R}^N$ y $|| \bullet ||^2$ es la norma inducida por el producto interno.
	\section{Función de ambigüedad cruzada}

	\begin{equation}
		\label{eq:caf}
		\psi(R,V) =
		\mathop{\int}_{-\tfrac{T}{2}}^{\tfrac{T}{2}}
		x_e(t) \cdot \,
		x_r^{\ast}\!\left(t - \frac{R}{c}\right) \cdot \,
		\exp\!\left(-j\,\frac{2\pi}{\lambda} V t\right)
		\,\mathrm{d}t
	\end{equation}
	
	\begin{equation}
		\label{eq:caf_dig}
		\psi(m,k) =
		\mathop{\sum}_{n = 0}^{T}
		x_e(t) \cdot \,
		x_r^{\ast}\!\left(n - m \right) \cdot \,
		\exp\!\left(-j\,\frac{2\pi}{N} k n\right)
	\end{equation}
	
		\subsection{Algoritmo Batch}
		
		
		\figura{.5}{./pr/caf.png}{Diagrama en bloque para calcular $\psi$ con el metodo Batch. }{Fuente: \cite{malanowski}}{\label{fig:batch}}
	\section{Detección de objetivos}
		\subsection{Ca - CFAR}
		Tradicionalmente, en los radares activos monostáticos, la detección se realizaba empleando una señal unidimensional, conocida como perfil de alcance. Este perfil se obtiene registrando la señal de eco tras la transmisión de un único pulso y aplicando posteriormente una compresión en rango mediante filtrado adaptado. Las muestras resultantes corresponden al tiempo rápido, el cual puede escalarse directamente con la distancia al radar.
		
		La Figura 6.3 presenta el diagrama en bloques de un sistema CFAR unidimensional. En cada iteración del algoritmo, el objetivo es determinar el umbral de detección asociado a la celda bajo prueba (Cell Under Test, CUT) a partir de las celdas de referencia o entrenamiento. Para ello, se considera un total de N$_w$ celdas de referencia a cada lado de la CUT. Esta, a su vez, se encuentra rodeada por N$_{g}$ celdas de guarda en cada extremo, las cuales no intervienen en la estimación del nivel de ruido. El uso de estas celdas de guarda se justifica porque el eco del blanco puede extenderse a lo largo de varias celdas de resolución —por efectos de la ventana de señal o del sobremuestreo—, lo que podría sesgar la estimación si no se evita su influencia directa. Por lo tanto, el número de celdas de guarda debe seleccionarse considerando el ancho esperado del eco recibido.
		
		La estimación del nivel de ruido a partir de las celdas de referencia puede realizarse de diversas maneras. Una de las más simples —y aún hoy entre las más empleadas— es el método Cell Averaging CFAR (CA-CFAR). En este enfoque, la estimación del nivel de ruido se obtiene calculando el valor medio de las muestras pertenecientes a ambas ventanas de referencia.
		
		\begin{equation}
			\hat{\sigma} = \dfrac{1}{2 \cdot N_w} \left(\hspace{5pt} \sum_{i = n - N_w - N_g}^{n - N_g - 1} x(i) \hspace{5pt} + \hspace{5pt} \sum_{i = n+N_g+1 }^{n + N_w + N_g}  x(i) \hspace{5pt}\right)
		\end{equation}
			
%	\chapter{Resultados}
	\section{Simulaciones}
	Para las simulaciones preeliminares y la verificacion de los algoritmos de procesamiento se usaron como señales ruido blanco complejo modelado como
	\begin{equation}
	x = (x_r + j x_i)*\sqrt{(N_0)} 	
	\end{equation}
	con $x_r$ y $x_i$ $\sim$ N(0,1). Para este tipo de señal, la funcion de ambiguedad cruzada resulta en una delta bidimensional. El escenario planteado es uno de 5 clutters  y un objetivo siguiendo la geometria de la figura [].
	
	
	\section{Adquisición y procesamiento de datos reales}
	\subsection{INSAP}
%	\include{capitulos/conclusiones}
	% References
	\cleardoublepage
	\addcontentsline{toc}{chapter}{Bibliografía}

	\printbibliography
	
	
	\end{document}